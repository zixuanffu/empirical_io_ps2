\documentclass[12pt]{article}[margin=1in]
\usepackage{fullpage,graphicx,psfrag,amsmath,amsfonts,verbatim}
\usepackage{multicol,multirow}
\usepackage[small,bf]{caption}
\usepackage{amsthm}
\usepackage{hyperref}
\usepackage{bbm} % for the indicator function to look good
\usepackage{color}
\usepackage{mathtools}
\usepackage{fancyhdr} % for the header
\usepackage{booktabs} % for regression table display (toprule, midrule, bottomrule)
\usepackage{adjustbox} % for regression table display
\usepackage{threeparttable} % to use table notes
\usepackage{natbib} % for bibliography
\input newcommand.tex
\bibliographystyle{apalike}
\setlength{\parindent}{0pt} % remove the automatic indentation

% Settings for page number in the footer
\pagestyle{fancy}
\fancyhf{}
\fancyfoot[C]{\thepage}
\renewcommand{\headrulewidth}{0pt}
\renewcommand{\footrulewidth}{0pt}

\title{\textbf{Dynamic Discrete Choice} \\
\vspace{.3cm}
\large Problem Set 2 \\
Empirical Industrial Organization}
\author{Zixuan, Anoosheh, Shuo}
\date{\today}

\begin{document}
\maketitle

\setcounter{page}{1}

The value function is given by
\begin{equation*}
    \begin{split}
        V(i,c,p,\epsilon_t) & = \max_{x \in \{0,1\}} \left\{ u(i,c,p,x) + \epsilon(x)+ \beta \E[V(i',c',p',\epsilon_{t+1})|i,c,p,x] \right\}     \\
                            & = \max_{x \in \{0,1\}} \big\{ u(i,c,p,x) + \epsilon(x)                                                             \\
                            & \quad + \beta \sum_{i',c',p'}\E_{\epsilon_{t+1}}[V(i',c',p',\epsilon_{t+1})|i,c,p,x] \Pr(i', c',p'|i,c,p,x) \big\}
    \end{split}
\end{equation*}
The utility function $u(i,c,p,x)$ is given by
\begin{equation*}
    u(i,c,p,x) = -\lambda \1(c>0)\1(i=0) + \alpha c -xp
\end{equation*}
In terms of the variables (data that we have),
\begin{itemize}
    \item $i$ is the inventory level.
    \item $c$ is the consumer's purchase decision (firm's sales).
    \item $p$ is the price.
    \item $x$ is the firm's purchase decision.
    \item $\epsilon(x)$ is choice specific random utility shock
\end{itemize}
In terms of the parameters (to be estimated, but actually given in this problem),
\begin{itemize}
    \item $\lambda=3$ is the penalty of stocking out (the consumer wants to buy, but the firm does not have the product).
    \item $\alpha=2$ is the marginal utility of selling the product.
    \item $\beta=0.99$ is the discount factor.
\end{itemize}

The variables follow a certain process. Here, we assume that the variables
follow discrete Markov process. The variables in the next period:
\begin{itemize}
    \item Inventory $i$ will be the current level + firm's purchase - sales: \begin{equation*}
              i'=\min \set{\bar{i}=4,i+x-c}
          \end{equation*}
    \item Consumer's purchase decision $c$ (firm's sales):\begin{equation*}
              c'=\begin{cases}
                  0 & \text{with probability } \gamma = 0.5   \\
                  1 & \text{with probability } 1-\gamma = 0.5
              \end{cases}
          \end{equation*}
    \item Price $p$ with two discrete states $p_s=0.5$ and $p_r=2$: \begin{equation*}
              \Pi=\begin{pmatrix}
                  0.75 & 0.25 \\
                  0.95 & 0.05
              \end{pmatrix}
          \end{equation*}
\end{itemize}

\section{Question 1: Transition Probability}
Since we have discrete state variables $(i,c,p)$, the transition probability
can be expressed in matrix form. Moreover, the transition of $c$ and $p$ are
independent of each other, $i$ and $x$. We only specify the transition of $i$
here which takes values from 0 to 4. \\ When $x = 0$,
\begin{equation}\label{eq:trans0}
    \begin{pmatrix}
        1   & 0   & 0   & 0   & 0   \\
        0.5 & 0.5 & 0   & 0   & 0   \\
        0   & 0.5 & 0.5 & 0   & 0   \\
        0   & 0   & 0.5 & 0.5 & 0   \\
        0   & 0   & 0   & 0.5 & 0.5
    \end{pmatrix}
\end{equation}
When $x = 1$,
\begin{equation}\label{eq:trans1}
    \begin{pmatrix}
        0.5 & 0.5 & 0   & 0   & 0   \\
        0   & 0.5 & 0.5 & 0   & 0   \\
        0   & 0   & 0.5 & 0.5 & 0   \\
        0   & 0   & 0   & 0.5 & 0.5 \\
        0   & 0   & 0   & 0   & 1
    \end{pmatrix}
\end{equation}
Then the transition probability matrix for state $s$ is given by the Kronecker product of the transition matrices of $i$, $c$, and $p$, which is $P_s(x) = P_i(x) \otimes P_c \otimes P_p$.
\section{Question 2: Expected Value Function}
\subsection{Expected/Intermediate Value Function $\bar{V}(i,c,p)$}
We denote $\bar{V}(s) = \E_{\epsilon}[V(s,\epsilon)]$ as the expected value function (I used to call it intermediate value function).
\begin{equation}\label{eq:exp_v}
    \begin{split}
        \bar{V}(s) & =\int \max_{x \in \{0,1\}} \big\{ u(s,x) + \epsilon(x)+ \beta \sum_{s'}\E_{\epsilon_{t+1}}[V(s',\epsilon_{t+1})|s,x] \Pr(s'|s,x)\big\}  f(\epsilon) d\epsilon \\
                   & =\int \max_{x \in \{0,1\}} \big\{ u(s,x) + \epsilon(x)+ \beta \sum_{s'}\bar{V}(s') \Pr(s'|s,x)\big\}  f(\epsilon) d\epsilon
    \end{split}
\end{equation}
Since $\epsilon$ is assumed to follow Type 1 Extreme Value distribution (which allows the integral over the maximum to collapse into a log-sum-exp form\footnote{I am not sure when and whether to include the Euler-Mascheroni constant $\gamma$}), we have
\begin{equation} \label{eq:exp_v_bin}
    \bar{V}(s) = \gamma+\ln\left(\sum_{x \in \{0,1\}} \exp\left\{ u(s,x) + \beta \sum_{s'}\bar{V}(s') \Pr(s'|s,x)\right\}\right)
\end{equation}

\subsection{Numerical solution of $\bar{V}(i,c,p)$}
We use the equation \ref{eq:exp_v_bin} to solve for $\bar{V}(i,c,p)$.  I rewrite the equations for all $s$ in matrix form. Since we have a total of $20=5\times 2\times 2$ discrete state $s$, denote
\begin{itemize}
    \item $\bar{V}$ as a vector of length 20.
    \item $u_0$ as a vector of length 20 where the $i$-th element is $u(s,0)$.
    \item $M_0$ as a matrix of size $20 \times 20$ where the $i$-th row is the vector of $\Pr(s'|s,0)$
\end{itemize}
Then we have
\begin{equation}\label{eq:exp_v_bin_matrix}
    \bar{V}= \gamma + \ln\pa{\exp\pa{u_0 + \beta M_0 \bar{V}} + \exp\pa{u_1 + \beta M_1 \bar{V}}}
\end{equation}
The goal is to numerically solve for equation \ref{eq:exp_v_bin_matrix} for $\bar{V}$. The result is shown in table \ref{tab:V_ss}.
\newpage
\begin{table}
    \centering
    % latex table generated in R 4.3.1 by xtable 1.8-4 package
% Wed Dec 18 14:39:36 2024
\begin{tabular}{rrrrr}
  \hline
 & Inventory & Consumer purchase & Price & Expected value function \\ 
  \hline
1 & 0.00 & 0.00 & 2.00 & 66.75 \\ 
  2 & 0.00 & 0.00 & 0.50 & 67.19 \\ 
  3 & 0.00 & 0.25 & 2.00 & 65.96 \\ 
  4 & 0.00 & 0.25 & 0.50 & 67.19 \\ 
  5 & 0.25 & 0.00 & 2.00 & 67.82 \\ 
  6 & 0.25 & 0.00 & 0.50 & 68.55 \\ 
  7 & 0.25 & 0.25 & 2.00 & 68.32 \\ 
  8 & 0.25 & 0.25 & 0.50 & 69.05 \\ 
  9 & 0.50 & 0.00 & 2.00 & 69.36 \\ 
  10 & 0.50 & 0.00 & 0.50 & 70.05 \\ 
  11 & 0.50 & 0.25 & 2.00 & 69.86 \\ 
  12 & 0.50 & 0.25 & 0.50 & 70.55 \\ 
  13 & 0.75 & 0.00 & 2.00 & 70.78 \\ 
  14 & 0.75 & 0.00 & 0.50 & 71.43 \\ 
  15 & 0.75 & 0.25 & 2.00 & 71.28 \\ 
  16 & 0.75 & 0.25 & 0.50 & 71.93 \\ 
  17 & 1.00 & 0.00 & 2.00 & 72.08 \\ 
  18 & 1.00 & 0.00 & 0.50 & 72.71 \\ 
  19 & 1.00 & 0.25 & 2.00 & 72.58 \\ 
  20 & 1.00 & 0.25 & 0.50 & 73.21 \\ 
  21 & 1.25 & 0.00 & 2.00 & 73.30 \\ 
  22 & 1.25 & 0.00 & 0.50 & 73.90 \\ 
  23 & 1.25 & 0.25 & 2.00 & 73.80 \\ 
  24 & 1.25 & 0.25 & 0.50 & 74.40 \\ 
  25 & 1.50 & 0.00 & 2.00 & 74.44 \\ 
  26 & 1.50 & 0.00 & 0.50 & 75.02 \\ 
  27 & 1.50 & 0.25 & 2.00 & 74.94 \\ 
  28 & 1.50 & 0.25 & 0.50 & 75.52 \\ 
  29 & 1.75 & 0.00 & 2.00 & 75.52 \\ 
  30 & 1.75 & 0.00 & 0.50 & 76.08 \\ 
  31 & 1.75 & 0.25 & 2.00 & 76.02 \\ 
  32 & 1.75 & 0.25 & 0.50 & 76.58 \\ 
  33 & 2.00 & 0.00 & 2.00 & 76.53 \\ 
  34 & 2.00 & 0.00 & 0.50 & 77.08 \\ 
  35 & 2.00 & 0.25 & 2.00 & 77.03 \\ 
  36 & 2.00 & 0.25 & 0.50 & 77.58 \\ 
  37 & 2.25 & 0.00 & 2.00 & 77.50 \\ 
  38 & 2.25 & 0.00 & 0.50 & 78.03 \\ 
  39 & 2.25 & 0.25 & 2.00 & 78.00 \\ 
  40 & 2.25 & 0.25 & 0.50 & 78.53 \\ 
  41 & 2.50 & 0.00 & 2.00 & 78.41 \\ 
  42 & 2.50 & 0.00 & 0.50 & 78.92 \\ 
  43 & 2.50 & 0.25 & 2.00 & 78.91 \\ 
  44 & 2.50 & 0.25 & 0.50 & 79.42 \\ 
  45 & 2.75 & 0.00 & 2.00 & 79.28 \\ 
  46 & 2.75 & 0.00 & 0.50 & 79.78 \\ 
  47 & 2.75 & 0.25 & 2.00 & 79.78 \\ 
  48 & 2.75 & 0.25 & 0.50 & 80.28 \\ 
  49 & 3.00 & 0.00 & 2.00 & 80.10 \\ 
  50 & 3.00 & 0.00 & 0.50 & 80.59 \\ 
  51 & 3.00 & 0.25 & 2.00 & 80.60 \\ 
  52 & 3.00 & 0.25 & 0.50 & 81.09 \\ 
  53 & 3.25 & 0.00 & 2.00 & 80.88 \\ 
  54 & 3.25 & 0.00 & 0.50 & 81.36 \\ 
  55 & 3.25 & 0.25 & 2.00 & 81.38 \\ 
  56 & 3.25 & 0.25 & 0.50 & 81.86 \\ 
  57 & 3.50 & 0.00 & 2.00 & 81.61 \\ 
  58 & 3.50 & 0.00 & 0.50 & 82.07 \\ 
  59 & 3.50 & 0.25 & 2.00 & 82.11 \\ 
  60 & 3.50 & 0.25 & 0.50 & 82.57 \\ 
  61 & 3.75 & 0.00 & 2.00 & 82.27 \\ 
  62 & 3.75 & 0.00 & 0.50 & 82.70 \\ 
  63 & 3.75 & 0.25 & 2.00 & 82.77 \\ 
  64 & 3.75 & 0.25 & 0.50 & 83.20 \\ 
  65 & 4.00 & 0.00 & 2.00 & 82.78 \\ 
  66 & 4.00 & 0.00 & 0.50 & 83.12 \\ 
  67 & 4.00 & 0.25 & 2.00 & 83.28 \\ 
  68 & 4.00 & 0.25 & 0.50 & 83.62 \\ 
   \hline
\end{tabular}

    \caption{Expected value function $\bar{V}(i,c,p)$ for each state $s=(i,c,p)$}
    \label{tab:V_ss}
\end{table}
\newpage
\section{Question 3: Simulation}
\begin{enumerate}
    \item At period $t=0$, simulate state $s$ as well as the shock $\epsilon \sim
              \text{EV}(1)$.
    \item Find the optimal choice $x$ given current $s$ and $\epsilon$ by the following
          \begin{equation*}
              x^* = \argmax_{x
                  \in \{0,1\}}  \left\{ u(s,x) + \epsilon+ \beta
              \sum_{s'}\bar{V}(s')\Pr(s'|s,x) \right\}
          \end{equation*}
    \item Given $x^*$, simulate a new state $s'$ from the transition matrix $M_{x^*}$.
    \item Repeat step 2 and 3 for $T$ periods.
\end{enumerate}
The  summary statistics are shown in the table \ref{tab:sim_des}.
\begin{enumerate}
    \item frequency of positive purchases,
    \item probability of purchasing on sale,
    \item average duration between sales,
    \item average duration between purchases
\end{enumerate}
\begin{table}
    \centering
    % latex table generated in R 4.3.1 by xtable 1.8-4 package
% Wed Dec 11 22:21:56 2024
\begin{tabular}{rlr}
  \hline
 & statistic & value \\ 
  \hline
1 & Frequency of purchase & 0.55 \\ 
  2 & Probability of purchase when sales & 0.62 \\ 
  3 & Average duration between sales & 1.27 \\ 
  4 & Average duration between purchases & 1.82 \\ 
   \hline
\end{tabular}

    \caption{Summary statistics of the simulation}
    \label{tab:sim_des}
\end{table}

\section{Question 4: Estimate $\bar{V}(i,c,p)$ using CCP method}
In this question, we \textbf{rewrite} equation \ref{eq:exp_v} in terms of the choice probability $\Pr(x|s)$.
\begin{equation}
    \begin{split}
        \bar{V}(s) & =\sum_{x \in \{0,1\}} \Pr(x|s) \big\{ u(s,x) + \E[\epsilon(x)|s,x] \\
                   & \quad + \beta \sum \bar{V}(s') \Pr(s'|s,x) \big\}
    \end{split}
\end{equation}
Note that the terms that are known are
\begin{itemize}
    \item $u(s,x)$ is the \textbf{utility function} which is explicitly given.
    \item $\Pr(s'|s,x)$ is the \textbf{transition probability} of $s$ given $s,x$ (see equation \ref{eq:trans0} and \ref{eq:trans1}).
\end{itemize}
The unknown terms are
\begin{itemize}
    \item $\Pr(x|s)$ is the \textbf{choice probability}.
    \item $E(\epsilon(x)|s,x)$ is the expectation of $\epsilon(x)$ conditional on $s$ and $x$ being the optimal choice. Under the assumption of T1EV, we have
          \begin{equation*}
              E(\epsilon(x)|s,x)=\gamma-\ln(\Pr(x|s))
          \end{equation*}
    \item $\bar{V}(s)$ is the \textbf{expected value function}.
\end{itemize}
In our binary choice case (with the usual assumption on $\epsilon$), we have
\begin{equation}\label{eq:exp_v_bin_ccp}
    \begin{split}
        \bar{V}(s) & =\sum_{x \in \{0,1\}} \Pr(x|s) \big\{ u(s,x) + \gamma-\ln(\Pr(x|s)) +\beta \sum \bar{V}(s') \Pr(s'|s,x) \big\} \\
                   & =\gamma + \Pr(x=0|s) \{u(s,0) + \beta \sum_{s'} \bar{V}(s') \Pr(s'|s,0)\}                                      \\
                   & \quad + \Pr(x=1|s) \{u(s,1) + \beta \sum_{s'} \bar{V}(s') \Pr(s'|s,1)\}                                        \\
    \end{split}
\end{equation}

The next steps are
\begin{enumerate}
    \item Estimate the choice probability ${\Pr}(x|s)$.
    \item Given the known $\Pr(s'|s,x)$ and $u(s,x)$, along with the estimated
          $\Pr(x|s)$, solve equation \ref{eq:exp_v_bin_ccp} for $\bar{V}(s)$.
\end{enumerate}
The results are tabulated against the $\bar{V}(s)$ estimated in question 2 in table \ref{tab:V_ss_ccp}.
\begin{table}
    \centering
    % latex table generated in R 4.3.1 by xtable 1.8-4 package
% Wed Dec 18 13:53:37 2024
\begin{tabular}{rrrrrr}
  \hline
 & Inventory & Consumer purchase & Price & $\bar{V}$ & $\bar{V}_{ccp}$ \\ 
  \hline
1 & 0.00 & 0.00 & 2.00 & 56.48 & 53.04 \\ 
  2 & 0.00 & 0.00 & 0.50 & 57.43 & 53.96 \\ 
  3 & 0.00 & 0.25 & 2.00 & 53.98 & 50.54 \\ 
  4 & 0.00 & 0.25 & 0.50 & 54.93 & 51.46 \\ 
  5 & 0.25 & 0.00 & 2.00 & 60.02 & 56.44 \\ 
  6 & 0.25 & 0.00 & 0.50 & 61.20 & 57.82 \\ 
  7 & 0.25 & 0.25 & 2.00 & 60.52 & 56.94 \\ 
  8 & 0.25 & 0.25 & 0.50 & 61.70 & 58.32 \\ 
  9 & 0.50 & 0.00 & 2.00 & 62.78 & 59.90 \\ 
  10 & 0.50 & 0.00 & 0.50 & 63.73 & 60.99 \\ 
  11 & 0.50 & 0.25 & 2.00 & 63.28 & 60.40 \\ 
  12 & 0.50 & 0.25 & 0.50 & 64.23 & 61.49 \\ 
  13 & 0.75 & 0.00 & 2.00 & 64.94 & 62.53 \\ 
  14 & 0.75 & 0.00 & 0.50 & 65.77 & 63.46 \\ 
  15 & 0.75 & 0.25 & 2.00 & 65.44 & 63.03 \\ 
  16 & 0.75 & 0.25 & 0.50 & 66.27 & 63.96 \\ 
  17 & 1.00 & 0.00 & 2.00 & 66.78 & 64.69 \\ 
  18 & 1.00 & 0.00 & 0.50 & 67.55 & 65.52 \\ 
  19 & 1.00 & 0.25 & 2.00 & 67.28 & 65.19 \\ 
  20 & 1.00 & 0.25 & 0.50 & 68.05 & 66.02 \\ 
  21 & 1.25 & 0.00 & 2.00 & 68.42 & 66.55 \\ 
  22 & 1.25 & 0.00 & 0.50 & 69.14 & 67.31 \\ 
  23 & 1.25 & 0.25 & 2.00 & 68.92 & 67.05 \\ 
  24 & 1.25 & 0.25 & 0.50 & 69.64 & 67.81 \\ 
  25 & 1.50 & 0.00 & 2.00 & 69.91 & 68.20 \\ 
  26 & 1.50 & 0.00 & 0.50 & 70.58 & 68.91 \\ 
  27 & 1.50 & 0.25 & 2.00 & 70.41 & 68.70 \\ 
  28 & 1.50 & 0.25 & 0.50 & 71.08 & 69.41 \\ 
  29 & 1.75 & 0.00 & 2.00 & 71.28 & 69.69 \\ 
  30 & 1.75 & 0.00 & 0.50 & 71.92 & 70.37 \\ 
  31 & 1.75 & 0.25 & 2.00 & 71.78 & 70.19 \\ 
  32 & 1.75 & 0.25 & 0.50 & 72.42 & 70.87 \\ 
  33 & 2.00 & 0.00 & 2.00 & 72.55 & 71.06 \\ 
  34 & 2.00 & 0.00 & 0.50 & 73.17 & 71.70 \\ 
  35 & 2.00 & 0.25 & 2.00 & 73.05 & 71.56 \\ 
  36 & 2.00 & 0.25 & 0.50 & 73.67 & 72.20 \\ 
  37 & 2.25 & 0.00 & 2.00 & 73.74 & 72.32 \\ 
  38 & 2.25 & 0.00 & 0.50 & 74.33 & 72.92 \\ 
  39 & 2.25 & 0.25 & 2.00 & 74.24 & 72.82 \\ 
  40 & 2.25 & 0.25 & 0.50 & 74.83 & 73.42 \\ 
  41 & 2.50 & 0.00 & 2.00 & 74.85 & 73.48 \\ 
  42 & 2.50 & 0.00 & 0.50 & 75.42 & 74.05 \\ 
  43 & 2.50 & 0.25 & 2.00 & 75.35 & 73.98 \\ 
  44 & 2.50 & 0.25 & 0.50 & 75.92 & 74.55 \\ 
  45 & 2.75 & 0.00 & 2.00 & 75.90 & 74.54 \\ 
  46 & 2.75 & 0.00 & 0.50 & 76.45 & 75.09 \\ 
  47 & 2.75 & 0.25 & 2.00 & 76.40 & 75.04 \\ 
  48 & 2.75 & 0.25 & 0.50 & 76.95 & 75.59 \\ 
  49 & 3.00 & 0.00 & 2.00 & 76.89 & 75.51 \\ 
  50 & 3.00 & 0.00 & 0.50 & 77.42 & 76.02 \\ 
  51 & 3.00 & 0.25 & 2.00 & 77.39 & 76.01 \\ 
  52 & 3.00 & 0.25 & 0.50 & 77.92 & 76.52 \\ 
  53 & 3.25 & 0.00 & 2.00 & 77.81 & 76.37 \\ 
  54 & 3.25 & 0.00 & 0.50 & 78.33 & 76.82 \\ 
  55 & 3.25 & 0.25 & 2.00 & 78.31 & 76.87 \\ 
  56 & 3.25 & 0.25 & 0.50 & 78.83 & 77.33 \\ 
  57 & 3.50 & 0.00 & 2.00 & 78.67 & 77.09 \\ 
  58 & 3.50 & 0.00 & 0.50 & 79.17 & 77.48 \\ 
  59 & 3.50 & 0.25 & 2.00 & 79.17 & 77.59 \\ 
  60 & 3.50 & 0.25 & 0.50 & 79.67 & 77.99 \\ 
  61 & 3.75 & 0.00 & 2.00 & 79.44 & 77.64 \\ 
  62 & 3.75 & 0.00 & 0.50 & 79.89 & 77.95 \\ 
  63 & 3.75 & 0.25 & 2.00 & 79.94 & 78.14 \\ 
  64 & 3.75 & 0.25 & 0.50 & 80.39 & 78.46 \\ 
  65 & 4.00 & 0.00 & 2.00 & 80.02 & 77.97 \\ 
  66 & 4.00 & 0.00 & 0.50 & 80.36 & 78.17 \\ 
  67 & 4.00 & 0.25 & 2.00 & 80.52 & 78.47 \\ 
  68 & 4.00 & 0.25 & 0.50 & 80.86 & 78.69 \\ 
   \hline
\end{tabular}

    \caption{Comparison of $\bar{V}(s)$ estimated by CCP method with the true value}
    \label{tab:V_ss_ccp}
\end{table}
The reason that second estimate is different/imprecise is lies in the fact that the choice probability $\Pr(x|s)$ is not precisely estimated. I have (naively) applied the frequency estimator. It may be improved by applying some other non-parametric estimator (along with some smoothing). There are no other sources of differences because all the $u(s,x)$ and transition probabilities are taken from the true model.
\end{document}