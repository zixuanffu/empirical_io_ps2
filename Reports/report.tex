\documentclass[12pt]{article}[margin=1in]
\usepackage{fullpage,graphicx,psfrag,amsmath,amsfonts,verbatim}
\usepackage{multicol,multirow}
\usepackage[small,bf]{caption}
\usepackage{amsthm}
\usepackage{hyperref}
\usepackage{bbm} % for the indicator function to look good
\usepackage{color}
\usepackage{mathtools}
\usepackage{fancyhdr} % for the header
\usepackage{booktabs} % for regression table display (toprule, midrule, bottomrule)
\usepackage{adjustbox} % for regression table display
\usepackage{threeparttable} % to use table notes
\usepackage{natbib} % for bibliography
\input newcommand.tex
\bibliographystyle{apalike}
\setlength{\parindent}{0pt} % remove the automatic indentation

% Settings for page number in the footer
\pagestyle{fancy}
\fancyhf{}
\fancyfoot[C]{\thepage}
\renewcommand{\headrulewidth}{0pt}
\renewcommand{\footrulewidth}{0pt}

\title{\textbf{Dynamic Discrete Choice} \\
\vspace{.3cm}
\large Problem Set 2 \\
Empirical Industrial Organization}
\author{Zixuan, Anoosheh, Shuo}
\date{\today}

\begin{document}
\maketitle

\setcounter{page}{1}

The value function is given by
\begin{equation*}
    \begin{split}
        V(i,c,p,\epsilon_t) & = \max_{x \in \{0,1\}} \left\{ u(i,c,p,x) + \epsilon(x)+ \beta \E[V(i',c',p',\epsilon_{t+1})|i,c,p,x] \right\}                                                        \\
                            & = \max_{x \in \{0,1\}} \left\{ u(i,c,p,x) + \epsilon(x)+ \beta \sum_{i',c',p'}\E_{\epsilon_{t+1}}[V(i',c',p',\epsilon_{t+1})|i,c,p,x] \Pr(i', c',p'|i,c,p,x) \right\} \\
    \end{split}
\end{equation*}
The utility function $u(i,c,p,x)$ is given by
\begin{equation*}
    u(i,c,p,x) = -\lambda \1(c>0)\1(i=0) + \alpha c -xp
\end{equation*}
In terms of the variables (data that we have),
\begin{itemize}
    \item $i$ is the inventory level.
    \item $c$ is the consumer's purchase decision (firm's sales).
    \item $p$ is the price.
    \item $x$ is the firm's purchase decision.
    \item $\epsilon(x)$ is choice specific random utility shock
\end{itemize}
In terms of the parameters (to be estimated, but actually given in this problem),
\begin{itemize}
    \item $\lambda=-3$ is the penalty of stocking out (the consumer wants to buy, but the firm does not have the product).
    \item $\alpha=2$ is the marginal utility of selling the product.
    \item $\beta=0.99$ is the discount factor.
\end{itemize}

The variables follow a certain process. Here, we assume that the variables
follow discrete Markov process. The variables in the next period:
\begin{itemize}
    \item Inventory $i$ will be the current level + firm's purchase - sales: \begin{equation*}
              i'=\min \set{\bar{i}=4,i+x-c}
          \end{equation*}
    \item Consumer's purchase decision $c$ (firm's sales):\begin{equation*}
              c'=\begin{cases}
                  0 & \text{with probability } \gamma = 0.5   \\
                  1 & \text{with probability } 1-\gamma = 0.5
              \end{cases}
          \end{equation*}
    \item Price $p$ with two discrete states $p_s=0.5$ and $p_r=2$: \begin{equation*}
              \Pi=\begin{pmatrix}
                  0.75 & 0.25 \\
                  0.95 & 0.05
              \end{pmatrix}
          \end{equation*}
\end{itemize}

\section{Question 1: Transition Probability}
Since we have discrete state variables $(i,c,p)$, the transition probability
can be expressed in matrix form. Moreover, the transition of $c$ and $p$ are
independent of each other, $i$ and $x$. We only specify the transition of $i$
here which takes values from 0 to 4. \\ When $x = 0$,
\begin{equation*}
    \begin{pmatrix}
        1   & 0   & 0   & 0   & 0   \\
        0.5 & 0.5 & 0   & 0   & 0   \\
        0   & 0.5 & 0.5 & 0   & 0   \\
        0   & 0   & 0.5 & 0.5 & 0   \\
        0   & 0   & 0   & 0.5 & 0.5
    \end{pmatrix}
\end{equation*}
When $x = 1$,
\begin{equation*}
    \begin{pmatrix}
        0.5 & 0.5 & 0   & 0   & 0   \\
        0   & 0.5 & 0.5 & 0   & 0   \\
        0   & 0   & 0.5 & 0.5 & 0   \\
        0   & 0   & 0   & 0.5 & 0.5 \\
        0   & 0   & 0   & 0   & 1
    \end{pmatrix}
\end{equation*}
Then the transition probability matrix for state $s$ is given by the Kronecker product of the transition matrices of $i$, $c$, and $p$, which is $P_s(x) = P_i(x) \otimes P_c \otimes P_p$.
\section{Question 2: Expected Value Function}
\subsection{Expected/Intermediate Value Function $\bar{V}(i,c,p)$}
We denote $\bar{V}(i,c,p) = \E_{\epsilon}[V(i,c,p,\epsilon)]$ as the expected value function (I used to call it intermediate value function).

\begin{equation*}
    \bar{V}(i,c,p) =\sum_{x \in \{0,1\}} P(x|i,c,p) \left\{ u(i,c,p,x) + \E[\epsilon(x)|i,c,p,x] + \beta \sum \bar{V}(i',c',p') \Pr(i',c',p'|i,c,p,x) \right\}
\end{equation*}
\subsection{Numerical solution of $\bar{V}(i,c,p)$}

\section{Question 3: Simulation}
\section{Question 4: Estimate $\bar{V}(i,c,p)$ using CCP method}
\pagebreak
\newpage

\end{document}