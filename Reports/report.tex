\documentclass[12pt]{article}[margin=1in]
\usepackage{fullpage,graphicx,psfrag,amsmath,amsfonts,verbatim}
\usepackage{multicol,multirow}
\usepackage[small,bf]{caption}
\usepackage{amsthm}
\usepackage{hyperref}
\usepackage{bbm} % for the indicator function to look good
\usepackage{color}
\usepackage{mathtools}
\usepackage{fancyhdr} % for the header
\usepackage{booktabs} % for regression table display (toprule, midrule, bottomrule)
\usepackage{adjustbox} % for regression table display
\usepackage{threeparttable} % to use table notes
\usepackage{natbib} % for bibliography
\input newcommand.tex
\bibliographystyle{apalike}
\setlength{\parindent}{0pt} % remove the automatic indentation

% Settings for page number in the footer
\pagestyle{fancy}
\fancyhf{}
\fancyfoot[C]{\thepage}
\renewcommand{\headrulewidth}{0pt}
\renewcommand{\footrulewidth}{0pt}

\title{\textbf{Dynamic Discrete Choice} \\
\vspace{.3cm}
\large Problem Set 2 \\
Empirical Industrial Organization}
\author{Zixuan, Anoosheh, Shuo}
\date{\today}

\begin{document}
\maketitle

\setcounter{page}{1}

The value function is given by
\begin{equation*}
    \begin{split}
        V(i,c,p,\epsilon_t) & = \max_{x \in \{0,1\}} \left\{ u(i,c,p,x) + \epsilon(x)+ \beta \E[V(i',c',p',\epsilon_{t+1})|i,c,p,x] \right\}                                                        \\
                            & = \max_{x \in \{0,1\}} \left\{ u(i,c,p,x) + \epsilon(x)+ \beta \sum_{i',c',p'}\E_{\epsilon_{t+1}}[V(i',c',p',\epsilon_{t+1})|i,c,p,x] \Pr(i', c',p'|i,c,p,x) \right\} \\
    \end{split}
\end{equation*}
The utility function $u(i,c,p,x)$ is given by
\begin{equation*}
    u(i,c,p,x) = -\lambda \1(c>0)\1(i=0) + \alpha c -xp
\end{equation*}
In terms of the variables (data that we have),
\begin{itemize}
    \item $i$ is the inventory level.
    \item $c$ is the consumer's purchase decision (firm's sales).
    \item $p$ is the price.
    \item $x$ is the firm's purchase decision.
    \item $\epsilon(x)$ is choice specific random utility shock
\end{itemize}
In terms of the parameters (to be estimated, but actually given in this problem),
\begin{itemize}
    \item $\lambda=3$ is the penalty of stocking out (the consumer wants to buy, but the firm does not have the product).
    \item $\alpha=2$ is the marginal utility of selling the product.
    \item $\beta=0.99$ is the discount factor.
\end{itemize}

The variables follow a certain process. Here, we assume that the variables
follow discrete Markov process. The variables in the next period:
\begin{itemize}
    \item Inventory $i$ will be the current level + firm's purchase - sales: \begin{equation*}
              i'=\min \set{\bar{i}=4,i+x-c}
          \end{equation*}
    \item Consumer's purchase decision $c$ (firm's sales):\begin{equation*}
              c'=\begin{cases}
                  0 & \text{with probability } \gamma = 0.5   \\
                  1 & \text{with probability } 1-\gamma = 0.5
              \end{cases}
          \end{equation*}
    \item Price $p$ with two discrete states $p_s=0.5$ and $p_r=2$: \begin{equation*}
              \Pi=\begin{pmatrix}
                  0.75 & 0.25 \\
                  0.95 & 0.05
              \end{pmatrix}
          \end{equation*}
\end{itemize}

\section{Question 1: Transition Probability}
Since we have discrete state variables $(i,c,p)$, the transition probability
can be expressed in matrix form. Moreover, the transition of $c$ and $p$ are
independent of each other, $i$ and $x$. We only specify the transition of $i$
here which takes values from 0 to 4. \\ When $x = 0$,
\begin{equation}\label{eq:trans0}
    \begin{pmatrix}
        1   & 0   & 0   & 0   & 0   \\
        0.5 & 0.5 & 0   & 0   & 0   \\
        0   & 0.5 & 0.5 & 0   & 0   \\
        0   & 0   & 0.5 & 0.5 & 0   \\
        0   & 0   & 0   & 0.5 & 0.5
    \end{pmatrix}
\end{equation}
When $x = 1$,
\begin{equation}\label{eq:trans1}
    \begin{pmatrix}
        0.5 & 0.5 & 0   & 0   & 0   \\
        0   & 0.5 & 0.5 & 0   & 0   \\
        0   & 0   & 0.5 & 0.5 & 0   \\
        0   & 0   & 0   & 0.5 & 0.5 \\
        0   & 0   & 0   & 0   & 1
    \end{pmatrix}
\end{equation}
Then the transition probability matrix for state $s$ is given by the Kronecker product of the transition matrices of $i$, $c$, and $p$, which is $P_s(x) = P_i(x) \otimes P_c \otimes P_p$.
\section{Question 2: Expected Value Function}
\subsection{Expected/Intermediate Value Function $\bar{V}(i,c,p)$}
We denote $\bar{V}(i,c,p) = \E_{\epsilon}[V(i,c,p,\epsilon)]$ as the expected value function (I used to call it intermediate value function).

\begin{equation}\label{eq:exp_v}
    \bar{V}(i,c,p) =\sum_{x \in \{0,1\}} P(x|i,c,p) \left\{ u(i,c,p,x) + \E[\epsilon(x)|i,c,p,x] + \beta \sum \bar{V}(i',c',p') \Pr(i',c',p'|i,c,p,x) \right\}
\end{equation}
Note that the terms that are known are
\begin{itemize}
    \item $u(i,c,p,x)$ is the \textbf{utility function} which is explicitly given.
    \item $\Pr(i',c',p'|i,c,p,x)$ is the \textbf{transition probability} of $i,c,p$ given $i,c,p,x$ (see equation \ref{eq:trans0} and \ref{eq:trans1}).
\end{itemize}
The unknown terms are
\begin{itemize}
    \item $P(x|i,c,p)$ is the \textbf{choice probability}.
    \item $E(\epsilon(x)|i,c,p,x)$ is the expectation of $\epsilon(x)$ conditional on $i,c,p$ and $x$ being the optimal choice.
    \item $\bar{V}(i,c,p)$ is the \textbf{expected value function}.

\end{itemize}
In the binary case with $\epsilon \sim T1EV$, instead of solving $V(s)$ as a function of $P(x|s)$ from the equation \ref{eq:exp_v}, we now have a simplified expression for $\bar{V}(s)$. \\
Let us denote ${v}(i,c,p,x) = u(i,c,p,x) +  \beta \sum \bar{V}(i',c',p') \Pr(i',c',p'|i,c,p,x)$. Then we have
\begin{equation}\label{eq:exp_v_binary}
    \begin{split}
        \bar{V}(s) & =  \gamma + \ln (1-P(x=1|s))                                                                                                        \\
                   & =  \gamma + \ln (\exp({v}(i,c,p,0)) + \exp({v}(i,c,p,1)))                                                                           \\
                   & =  \gamma + \ln\pa{\exp\pa{u(s,0) +  \beta \sum_{s'} \bar{V}(s') \Pr(s'|s,0)} + \exp\pa{u(s,1) +  \beta \sum_{s'} \bar{V}(s') \Pr(s
                '|s,1)}}
    \end{split}
\end{equation}
Or should I write it as
\begin{equation}
    \begin{split}
        \bar{V}(s) & =  \gamma + \ln (1-P(x=1|s))                                                                                                      \\
                   & =  \gamma + \ln (1+\exp({v}(i,c,p,1)-{v}(i,c,p,0)))                                                                               \\
                   & =  \gamma + \ln\pa{1+ \exp(u(s,1) - u(s,0) +  \beta \sum_{s'} \bar{V}(s') \Pr(s'|s,1) - \beta \sum_{s'} \bar{V}(s') \Pr(s'|s,0))}
    \end{split}
\end{equation}
\subsection{Numerical solution of $\bar{V}(i,c,p)$}
We use the equation \ref{eq:exp_v_binary} to solve for $\bar{V}(i,c,p)$ numerically.  I want rewrite the equations for all $s$ in matrix form. Since we have a total of $20=5\times 2\times 2$ discrete state $s$, denote
\begin{itemize}
    \item $\bar{V}$ as a vector of length 20.
    \item $u_0$ as a vector of length 20 where the $i$-th element is $u(s,0)$.
    \item $M_0$ as a matrix of size $20 \times 20$ where the $i$-th row is the vector of $\Pr(s'|s,0)$
\end{itemize}
Then we have

\begin{equation}\label{eq:exp_v_binary_matrix}
    \bar{V}= \gamma + \ln\pa{\exp\pa{u_0 + \beta M_0 \bar{V}} + \exp\pa{u_1 + \beta M_1 \bar{V}}}
\end{equation}
The goal is to numerically solve for this equation \ref{eq:exp_v_binary_matrix}
for $\bar{V}$. The result is shown in the table below.
\begin{table} \fontsize{10pt}{12pt}\selectfont
    \centering
    % latex table generated in R 4.3.1 by xtable 1.8-4 package
% Wed Dec 18 14:39:36 2024
\begin{tabular}{rrrrr}
  \hline
 & Inventory & Consumer purchase & Price & Expected value function \\ 
  \hline
1 & 0.00 & 0.00 & 2.00 & 66.75 \\ 
  2 & 0.00 & 0.00 & 0.50 & 67.19 \\ 
  3 & 0.00 & 0.25 & 2.00 & 65.96 \\ 
  4 & 0.00 & 0.25 & 0.50 & 67.19 \\ 
  5 & 0.25 & 0.00 & 2.00 & 67.82 \\ 
  6 & 0.25 & 0.00 & 0.50 & 68.55 \\ 
  7 & 0.25 & 0.25 & 2.00 & 68.32 \\ 
  8 & 0.25 & 0.25 & 0.50 & 69.05 \\ 
  9 & 0.50 & 0.00 & 2.00 & 69.36 \\ 
  10 & 0.50 & 0.00 & 0.50 & 70.05 \\ 
  11 & 0.50 & 0.25 & 2.00 & 69.86 \\ 
  12 & 0.50 & 0.25 & 0.50 & 70.55 \\ 
  13 & 0.75 & 0.00 & 2.00 & 70.78 \\ 
  14 & 0.75 & 0.00 & 0.50 & 71.43 \\ 
  15 & 0.75 & 0.25 & 2.00 & 71.28 \\ 
  16 & 0.75 & 0.25 & 0.50 & 71.93 \\ 
  17 & 1.00 & 0.00 & 2.00 & 72.08 \\ 
  18 & 1.00 & 0.00 & 0.50 & 72.71 \\ 
  19 & 1.00 & 0.25 & 2.00 & 72.58 \\ 
  20 & 1.00 & 0.25 & 0.50 & 73.21 \\ 
  21 & 1.25 & 0.00 & 2.00 & 73.30 \\ 
  22 & 1.25 & 0.00 & 0.50 & 73.90 \\ 
  23 & 1.25 & 0.25 & 2.00 & 73.80 \\ 
  24 & 1.25 & 0.25 & 0.50 & 74.40 \\ 
  25 & 1.50 & 0.00 & 2.00 & 74.44 \\ 
  26 & 1.50 & 0.00 & 0.50 & 75.02 \\ 
  27 & 1.50 & 0.25 & 2.00 & 74.94 \\ 
  28 & 1.50 & 0.25 & 0.50 & 75.52 \\ 
  29 & 1.75 & 0.00 & 2.00 & 75.52 \\ 
  30 & 1.75 & 0.00 & 0.50 & 76.08 \\ 
  31 & 1.75 & 0.25 & 2.00 & 76.02 \\ 
  32 & 1.75 & 0.25 & 0.50 & 76.58 \\ 
  33 & 2.00 & 0.00 & 2.00 & 76.53 \\ 
  34 & 2.00 & 0.00 & 0.50 & 77.08 \\ 
  35 & 2.00 & 0.25 & 2.00 & 77.03 \\ 
  36 & 2.00 & 0.25 & 0.50 & 77.58 \\ 
  37 & 2.25 & 0.00 & 2.00 & 77.50 \\ 
  38 & 2.25 & 0.00 & 0.50 & 78.03 \\ 
  39 & 2.25 & 0.25 & 2.00 & 78.00 \\ 
  40 & 2.25 & 0.25 & 0.50 & 78.53 \\ 
  41 & 2.50 & 0.00 & 2.00 & 78.41 \\ 
  42 & 2.50 & 0.00 & 0.50 & 78.92 \\ 
  43 & 2.50 & 0.25 & 2.00 & 78.91 \\ 
  44 & 2.50 & 0.25 & 0.50 & 79.42 \\ 
  45 & 2.75 & 0.00 & 2.00 & 79.28 \\ 
  46 & 2.75 & 0.00 & 0.50 & 79.78 \\ 
  47 & 2.75 & 0.25 & 2.00 & 79.78 \\ 
  48 & 2.75 & 0.25 & 0.50 & 80.28 \\ 
  49 & 3.00 & 0.00 & 2.00 & 80.10 \\ 
  50 & 3.00 & 0.00 & 0.50 & 80.59 \\ 
  51 & 3.00 & 0.25 & 2.00 & 80.60 \\ 
  52 & 3.00 & 0.25 & 0.50 & 81.09 \\ 
  53 & 3.25 & 0.00 & 2.00 & 80.88 \\ 
  54 & 3.25 & 0.00 & 0.50 & 81.36 \\ 
  55 & 3.25 & 0.25 & 2.00 & 81.38 \\ 
  56 & 3.25 & 0.25 & 0.50 & 81.86 \\ 
  57 & 3.50 & 0.00 & 2.00 & 81.61 \\ 
  58 & 3.50 & 0.00 & 0.50 & 82.07 \\ 
  59 & 3.50 & 0.25 & 2.00 & 82.11 \\ 
  60 & 3.50 & 0.25 & 0.50 & 82.57 \\ 
  61 & 3.75 & 0.00 & 2.00 & 82.27 \\ 
  62 & 3.75 & 0.00 & 0.50 & 82.70 \\ 
  63 & 3.75 & 0.25 & 2.00 & 82.77 \\ 
  64 & 3.75 & 0.25 & 0.50 & 83.20 \\ 
  65 & 4.00 & 0.00 & 2.00 & 82.78 \\ 
  66 & 4.00 & 0.00 & 0.50 & 83.12 \\ 
  67 & 4.00 & 0.25 & 2.00 & 83.28 \\ 
  68 & 4.00 & 0.25 & 0.50 & 83.62 \\ 
   \hline
\end{tabular}

\end{table}
\section{Question 3: Simulation}
\begin{enumerate}
    \item At period $t=0$, simulate state $s$ as well as the shock $\epsilon \sim T1EV$.
    \item Find the optimal choice $x$ given current $s$ and $\epsilon$ by the following \begin{equation*}
              x^* = \argmax_{x
                  \in \{0,1\}}  \left\{ u(s,x) + \epsilon+ \beta
              \sum_{s'}\bar{V}(s')\Pr(s'|s,x) \right\}
          \end{equation*}
    \item Given $x^*$, simulate a new state $s'$ from the transition matrix $M_{x^*}$.
    \item Repeat step 2 and 3 for $T$ periods.
\end{enumerate}

\section{Question 4: Estimate $\bar{V}(i,c,p)$ using CCP method}
In this question we focus on the first line of equation \ref{eq:exp_v_binary} to estimate the $\bar{V}(i,c,p)$. We first estimate the choice probability $\hat{P}(x|s)$ and then recover $\hat{\bar{V}}(s)$. That is
\begin{equation*}
    \hat{\bar{V}}(s) =  \gamma + \ln (1-\hat{P}(x=1|s))
\end{equation*}
\pagebreak
\newpage

\end{document}